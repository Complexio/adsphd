% !TeX root = ../../thesis.tex
\chapter[Modal mineralogy]{Modal mineralogy\footnote{This chapter is in part based on \fullcite{Weltje_Paredis_2020}}}\label{ch:modal_mineralogy}
Modal mineralogy, also known as the volumetric proportions of minerals, within a parent rock is used as the measure for the first fundamental property of composition according to the \gls{GGD}.
It defines the amount of each mineral that is present within the parent rock as a volumetric proportion. %
The modal mineralogy can be estimated through optical microscopy and is hence a straightforward property to establish. %

What might be less straightforward is how representative the modal mineralogy of a certain sample (i.e., thin section) is for the entire pluton. %
To answer that question, part of the research conducted in this PhD, looked at the variability of the modal mineralogy within plutons. %
This may not only help us to assess the variability with regards to the proposed research question but might also aid in a better error-propagation calculation of the final SedGen model. %

\section{First building block / fundamental property}

\section{Spatial variation of modal mineralogy}







%%%%%%%%%%%%%%%%%%%%%%%%%%%%%%%%%%%%%%%%%%%%%%%%%%
% Keep the following \cleardoublepage at the end of this file,
% otherwise \includeonly includes empty pages.
\cleardoublepage

% vim: tw=70 nocindent expandtab foldmethod=marker foldmarker={{{}{,}{}}}
