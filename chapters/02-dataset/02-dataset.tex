% !TeX root = ../../thesis.tex
\chapter{Dataset}\label{ch:dataset}

\section{Chapter contents}
\begin{itemize}
    \item Introduction to main data set of Heins (1992): Geological setting of different plutons, number and identity of samples, etc. . %
    \item Introduction to three main ‘building blocks’ of modal mineralogy, relative crystal interfaces and CSD
\end{itemize}

\section{Data from literature}
    \subsection{Data from Heins (1992)}
        The PhD thesis of \Textcite{Heins_1992} was the main source for data of this project. %
        Since it holds detailed data on the mineralogical composition and texture of six granitoid plutons it was very suited to serve as the starting point for this PhD. %
        Later on, the available data would be extended by means of own data collection (see next section). %

        \subsubsection{Granitoid plutons}
        Six plutons were sampled by \Textcite{Heins_1992} which resulted in \ca 50 thin sections of parent rock and \ca 150 sediment samples across three size fractions (see Table \ref{tab:n_samples_Heins}). %

            \begin{table}[t]
    \centering
    \caption{Number of samples collected by \Textcite{Heins_1992} for parent rock (P) and sediment (C: coarse; M: medium; F: fine)}
    \label{tab:n_samples_Heins}
    \begin{threeparttable}
        \begin{tabular}{lllll}
            \toprule
                  & P  & C & M & F \\
            \midrule
            AZ    & 15 &   &   &   \\
            CA-NS & 5  &   &   &   \\
            CA-EW & 6  &   &   &   \\
            GR    & 10 &   &   &   \\
            MT    & 8  &   &   &   \\
            WA    & 4  &   &   &   \\
            \bottomrule
        \end{tabular}

        % \begin{tablenotes}
        % \end{tablenotes}
    \end{threeparttable}
\end{table}


    \subsection{Additional data sources}
        Throughout the PhD, other sources for data of various forms has also been used. %
        One worth noting here is the \Textcite{Vistelius_et_al_1983} which was used for both the spatial variability of modal mineralogy and textural information studies. %
        References to those may be found in the relevant chapters. %

\section{Own data}
    Additional data was gathered to elaborate the dataset already collected by \Textcite{Heins_1992}. %
    The main addition was the digital image analysis carried out over a period of four months. %
    During this analysis, data on crystal size was collected of all available thin sections from \Textcite{Heins_1992}. %



%%%%%%%%%%%%%%%%%%%%%%%%%%%%%%%%%%%%%%%%%%%%%%%%%%
% Keep the following \cleardoublepage at the end of this file,
% otherwise \includeonly includes empty pages.
\cleardoublepage

% vim: tw=70 nocindent expandtab foldmethod=marker foldmarker={{{}{,}{}}}
