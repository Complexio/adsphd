% !TeX root = ../../thesis.tex
\chapter{Boundary conditions}\label{ch:boundary_conditions}

\section{Chapter contents}
\begin{itemize}
    \item Climate (temperature, pH, precipitation)
    \item Physiography (relief)
    \item Weathering rates (time vs intensity) \(\rightarrow\) Alex Blum (reference)

\end{itemize}

\section{Climate}
    Climatic conditions also have a role to play in the sediment generation process. %
    \textbf{Temperature} and \textbf{precipitation} are the two most obvious factors.
    The \textbf{pH} of water can also be of importance when chemical weathering is the main form of weathering (e.g. solubility of K-feldspar). %

\section{Physiography}
    The \textbf{relief} of the terrain in which the sediment generation is occurring must also be taken into consideration. %
    Along with the present \textbf{vegetation}, both will have an effect on mechanical and chemical weathering. %

\section{Weathering rate}
    All the factors mentioned in the above sections boil down to have a combined effect on the \textbf{rate of weathering}, be it mechanical or chemical. %


%%%%%%%%%%%%%%%%%%%%%%%%%%%%%%%%%%%%%%%%%%%%%%%%%%
% Keep the following \cleardoublepage at the end of this file,
% otherwise \includeonly includes empty pages.
\cleardoublepage

% vim: tw=70 nocindent expandtab foldmethod=marker foldmarker={{{}{,}{}}}
