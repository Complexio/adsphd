% !TeX root = ../../thesis.tex
\chapter[Interfacial composition]{Interfacial composition\footnote{This chapter is in part based on \fullcite{Weltje_et_al_2018}}}\label{ch:interfaces}
The second property of the \gls{GGD} is that of texture which thus tells something about the spatial relations between elements (i.e., minerals) of an aggregate (i.e., parent rock). %
The measure used during this PhD to characterize this property was the interfacial composition which looks at the volumetric proportions of element contacts/interfaces between the specified classes. %
In our case these elements consist of crystals while the classes are the different mineral classes but it can also match other elements/classes as is discussed in \textcite{LePera_Morrone_2020}. %



\section{Second building block / fundamental property}
\section{}




%%%%%%%%%%%%%%%%%%%%%%%%%%%%%%%%%%%%%%%%%%%%%%%%%%
% Keep the following \cleardoublepage at the end of this file,
% otherwise \includeonly includes empty pages.
\cleardoublepage

% vim: tw=70 nocindent expandtab foldmethod=marker foldmarker={{{}{,}{}}}
