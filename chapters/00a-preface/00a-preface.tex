% !TeX root = ../../thesis.tex
\chapter*{Preface}\label{ch:preface}

``Closing the gap in source-to-sink modelling'' sounded as a real challenge to achieve in just four years of a PhD. %
Now, five and half years later, I can honestly say that couldn't be less true. %
Although the challenge is not yet complete, a first pioneering move into making this preposition sound has been made. %
This move is explained and discussed throughout this thesis alongside with the documentation of the resulting software package. %

It was a quantitative approach to sediment generation modelling that was needed to close this aforementioned gap. %
Although there have been some pioneers before me, there are few, and we had basically to start from scratch, were it not for Gert Jan's research proposals with pseudo code. %
Coupled will all of the experience of Gert Jan and my own acquired programming skills I was able to develop a Python package called \textit{SedGen}. %

Allow me to convey how Gert Jan convinced to take on this challenge:
He told me that we could probably not say a lot of absolute measures about our results of the sediment generation modelling because we would use relative properties to start with as input. %
Nevertheless, relative things can already allow one to say something about the outcome. %
He illustrated this with an example I still remember vividly: Say it's Easter day and there is a certain amount of eggs hidden in the garden. %
You do not know the exact number of eggs, nor do you know where they were hidden. %
Imagine that there are two children that will try to find and collect the eggs. %
One of them is four years old and the other is eight. %
Asked with how many eggs each of the children will be able to find, you cannot possibly give the correct answer, except for mere coincidence. %
But if you are asked, which of the children will probably be able to collect most of the eggs, things get interesting. %
You could take their age in consideration and base your answer on that, or you could try to gain more information about the physical traits of the children in advance. %
All to say that by taking certain properties into consideration, you will be able to give a probabilistic answer to a relative question rather than a guess to a absolute question. %

This PhD gave me the opportunity to venture into the exciting world of programming and establish a strong foothold in Python. %

I would like to thank my colleagues and friends from the Geo-institute. %
Nick for the good times we had organizing the PhD Midweek, Geology BBQ and acting as Sinterklass and Easter bunny. Strangely enough driving to the supermarket with 6000+ RPM at ten minutes before closing time to basically `rob` the store's meat supply to make sure enough meat can be put on the barbecue is one of my fondest memories. %%
Aside from that I also enjoyed all the discussions about Python and data science we had, and big thanks for helping me getting around in the first year of my PhD. %
Bura and Alex for the friendship and many talks. %
Niels, Dominique, Jeroen, Hannes, Rieko, Sofie, Bura, and Alex for the numerous table tennis occasions. %
Alex, Michaël, Dominque, Wei Wei, Sander W., Fernando, Anneleen, Rik, Sander M., Wei Wang, Ward, Nick, Johanna, Hannes, Dongyu, Niels, Sreçko, Thomas, Wouter and many others that joined only once or twice for one or more of the fifteen LAN parties you joined each time I organized one. %
Michaël, Johanna, Sander, Monika for the summers filled with games of Kubb. %
Satur for her advice in the later stages of my PhD and the nice talks in our office. %


\begin{flushright}
    Bram Paredis \\
    Herent, August 2021
\end{flushright}

\instructionspreface


%%%%%%%%%%%%%%%%%%%%%%%%%%%%%%%%%%%%%%%%%%%%%%%%%%
% Keep the following \cleardoublepage at the end of this file,
% otherwise \includeonly includes empty pages.
\cleardoublepage

% vim: tw=70 nocindent expandtab foldmethod=marker foldmarker={{{}{,}{}}}
