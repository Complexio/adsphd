% !TeX root = ../../thesis.tex
\chapter{SedGen: design and architecture}\label{ch:sedgen_architecture}

%
% Look at Hobley et al 2017 for a good overview of how to write the backbones for this chapter
%

\section{Chapter contents}
\begin{itemize}
    \item General overview of how data should be formatted to be accepted by the model and the modules at work to clean, transform and calculate data in the case of legacy data.
    \item Types of weathering and their interaction
    \item Types of crystals/grains within SedGen (poly, mono, RF…)
    \item Discuss their implementation into the SedGen model and how they are tracked
    \item Different frameworks that are used in SedGen (e.g. Lagrangian framework, compositional space vs physical space)
    \item Discuss data that SedGen model can output and at what timesteps (GSD, mineralogy, grain number frequencies)

\end{itemize}

\section{Design guidelines}
SedGen should be written in a popular programming language. %
The code base should be easily maintainable, while also being in some sort of version control system. %
A modular structure of the code base is also preferable. %


\section{Programming language}
SedGen is written in Python and uses as main packages: numpy, pandas, matplotlib, etc. . %
All of these packages are widely used throughout the scientific community. % (ref to black hole picture?).
\begin{itemize}
    \item Python is easy to use
    \item Python is low-level
\end{itemize}



%%%%%%%%%%%%%%%%%%%%%%%%%%%%%%%%%%%%%%%%%%%%%%%%%%
% Keep the following \cleardoublepage at the end of this file,
% otherwise \includeonly includes empty pages.
\cleardoublepage

% vim: tw=70 nocindent expandtab foldmethod=marker foldmarker={{{}{,}{}}}
