% !TeX root = ../../thesis.tex
\chapter{SedGen: design and architecture}\label{ch:sedgen_architecture}

%
% Look at Hobley et al 2017 for a good overview of how to write the backbones for this chapter
%
\section{Introduction}
SedGen was written in Python, a popular, easy-to-use, programming language. %
The code base has been tracked through Git as version control system (VCS) with an online version available on GitHub (see link). %
This makes sure SedGen can be easily maintained in the future as well as receive updates. %
It was opted to write the different functions of SedGen into various modules thus resulting in a modular structure. %
As long as the input and output formats cohere to the expected formats, any function can be implemented to modify the rock along its way to sediment. %
How meaningful such operations might be, however, is totally dependent on the understanding of the function's creator. %


\section{Python}
Python has been developed by Guido van Rossem in 1991 to overcome problems of available programming languages at the time. %
It has been written as a dynamic language, meaning that the user does not need to specify things such as data types, memory usage etc. . %
Since then it has grown a large community base, resulting in popularity among various fields such as general programming, data science, web development, etc. . %
Although, in the past Python 2 had the largest use cases, after its deprecation in 2020 Python 3 is now the way forward. %
Therefore, SedGen was also implemented in Python 3 with a vision towards future-proofing it. %

Aside from Python`s standard library of packages some additional third party packages were used, including:
\begin{itemize}
    \item \textit{Numpy} for linear algebra and mathematical operations in general. %
    \item \textit{Numba} to compile Numpy functions before actually running SedGen, thus making Python behave as a static language resulting in significant execution time speed-ups. %
    \item \textit{Scipy} to use mathematical distributions and various additions to Numpy's library. %
    \item \textit{Pandas} to handle all tabular format data and handling input/output operations from/to spreadsheets. %
    \item \textit{Matplotlib} and \textit{Seaborn} to plot and `prettify' results.
\end{itemize}
All of these packages are widely used throughout the scientific community, of which a nice example is their use to take the first images of a black hole (ref). %


\section{SedGen's initialization}
The user is expected to provide the following model properties as a necessity:
\begin{itemize}
    \item Modal mineralogy to specify which minerals are present in the rock as well as their volumetric proportions. %
    \item Mean and standard deviation of each mineral`s crystal size distribution, which, as discussed in section ..., is assumed to be derived from a (truncated) log-normal distribution. %
\end{itemize}

As optional properties the following may be provided during the initialization:
\begin{itemize}
    \item Interfacial composition to specify the texture of the parent rock. %
    If this property is not specified, the alpha factor can be specified to determine the interfacial composition after the number frequencies of the parent rock have been calculated. %
    In the event when the alpha factor is also not given, a default alpha factor of 1 will be used thus resulting in a random texture or a pre-specified factor for a certain rock type with a certain texture based on literature included into SedGen beforehand; (e.g., 1.05 for granite). %
    \item Parent rock volume to specify the initial volume of rock to be weathered.
    By default this is set to \SI{1}{\cubic\metre} of parent rock.
    \item Boundary conditions such as climate, including temperature and precipitation, and vegetation might be given or be chosen from a pre-specified list of options. %
    Another possibility exists in providing an option per time step in the model, thus altering the ratio between mechanical and chemical weathering throughout the model's runtime. %
\end{itemize}

These initial parameters are then passed to a constructor instance of the SedGen class. %
The initialization of an instance takes about \SI{10}{\second} (for 10 million crystals initially). %
After assigning the initial parameters as attributes of the instance, new attributes are calculated based on them. %
Firstly the binning occurs, in which the crystal sizes are discretizized for computational benefits. %
Secondly, the mineral occurrences are simulated based on the modal mineralogy and the crystal-size distribution data. %
    This is performed in a separate function in which the parent rock volume is filled up with crystals until it is filled. %
    This is done per mineral class by taking the product of the modal mineralogy and parent rock volume as the mineral class' volume. %
    By initializing a log-normal distribution based on the csd mean and std, crystal sizes might be generated from it.
    A `learning-rate' parameter is passed to the function to determine how many crystals are requested from the log-normal distribution during every iteration of the while loop. %
    The while loop continues until the sum of the volume of generated crystals reaches the mineral class' volume. %
    By tweaking the learning rate the resulting simulated mineral class' volume might better approximate the true mineral class' volume. %
    Once the required volume has been filled, the array of generated crystal volumes is binned according to the already initialized volume bin array. %

At this point, the crystal size/volume array has been initialized. %
The number of crystals per mineral class is also known at this point. %
These numbers might be used later on to request new material to be generated to enter the model at a later timestep. %
This generation of material can then occur without the need of filling up a volume via a learning-curve but might be performed by just requesting N crystals from the log-normal crystal size distribution. %



\section{Binning}
Since SedGen was developed for use on personal computers, trade-offs between accuracy and performance had to be made. %
One such trade-off is the use of bins to represent crystal sizes. %
By default, the model works with 1500 crystal size classes in a geometric series of \SIrange[parse-numbers=false]{2^{-10}}{2^{5}}{\milli\metre}. %\(2^{-10}\) to \(2^{5}\). %
This translates to a minimal crystal size of \ca \SI{1}{\micro\metre} and a maximal crystal size of \SI{30}{\milli\metre}. %

In most of the model's functions, however, volume classes are used for calculations. %
This is done to establish a `volume-balance' cfr. mass-balance practices. %
The size bins are converted to volume bins under the assumption that all crystal are spheres. %
Hence: %
    \begin{equation}
        bins_{V} = \frac{4}{3} \pi\ bins_{S}^{3}
    \end{equation}
The bins are represented by bin labels ranging from 0 up to 1500 in the default case. %
For ease of use, the median of each bin is used during calculations. %
    \begin{equation}
        bins_{S, med} = median(bins_{S})
    \end{equation}
    \begin{equation}
        bins_{V, med} = median(bins_{V})
    \end{equation}
% Explain search_bins and ratio_search_bins here?


\section{Crystals}
The base unit of the model is the crystal which represents a single entity of a mineral class. %
Its boundaries thus lie in the crystallographic characteristics of the mineral class to which it belongs. %
A distinction between so-called sub-crystals is not made within the model even though crystals may split in two during intra-crystal breakage. %
If such a splitting happens the crystal (or mono-crystalline grain at this point) is merely assigned to two new crystals of the same mineral class, be it of a smaller size/volume of course. %
A mineral class can represent one mineral or several ones, e.g. for accessory minerals, within the parent rock that is being studied. %
This should be borne in mind when certain physical or chemical properties are set for these `combined mineral classes'. %



\section{Grain representation}
SedGen makes a distinction between mono- and poly-crystalline grains. %
The former is a grain consisting out of one crystal while the latter may consist out of two or more crystals. %
Although a crystal might consist of sub-crystals these are not considered in SedGen. %

The poly-crystalline grains are represented by a list of arrays in which each array represents one pcg.
\begin{equation}
    pcgs = [pcg_1, pcg_2, \ldots, pcg_p]
\end{equation}
The mono-crystalline grains are represented by a n x m x d matrix in which the values represent the number of mcgs (of a specific chemical weathering state, mineral class and size) present in the model. %

\begin{figure}
    \centering
    % !TeX root = ../../thesis.tex
\begin{tikzpicture}[auto matrix/.style={matrix of nodes,
  draw,thick,inner sep=0pt,
  nodes in empty cells,column sep=-0.2pt,row sep=-0.2pt,
  cells={nodes={minimum width=1.9em,minimum height=1.9em,
   draw,very thin,anchor=center,fill=white,
   % execute at begin node={%
   % $\vphantom{x_|}\ifnum\the\pgfmatrixcurrentrow<4
   %   \ifnum\the\pgfmatrixcurrentcolumn<4
   %    {#1}^{\the\pgfmatrixcurrentrow}_{\the\pgfmatrixcurrentcolumn}
   %   \else
   %    \ifnum\the\pgfmatrixcurrentcolumn=5
   %     {#1}^{\the\pgfmatrixcurrentrow}_{N}
   %    \fi
   %   \fi
   %  \else
   %   \ifnum\the\pgfmatrixcurrentrow=5
   %    \ifnum\the\pgfmatrixcurrentcolumn<4
   %     {#1}^{T}_{\the\pgfmatrixcurrentcolumn}
   %    \else
   %     \ifnum\the\pgfmatrixcurrentcolumn=5
   %      {#1}^{T}_{N}
   %     \fi
   %    \fi
   %   \fi
   %  \fi
   %  \ifnum\the\pgfmatrixcurrentrow\the\pgfmatrixcurrentcolumn=14
   %   \cdots
   %  \fi
   %  \ifnum\the\pgfmatrixcurrentrow\the\pgfmatrixcurrentcolumn=41
   %   \vdots
   %  \fi
   %  \ifnum\the\pgfmatrixcurrentrow\the\pgfmatrixcurrentcolumn=44
   %   \ddots
   %  \fi$
   %  }
  }}}]
 \matrix[auto matrix=z,xshift=3em,yshift=3em](matz){
  0 & 0 & 0 & 0 & 0 \\
  0 & 0 & 0 & 0 & 0 \\
  0 & 0 & 0 & 0 & 0 \\
  0 & 0 & 0 & $\ddots$ & $\vdots$ \\
  0 & 0 & 0 & $\cdots$ & 0 \\
 };
 \matrix[auto matrix=y,xshift=1.5em,yshift=1.5em](maty){
  0 & 0 & 0 & 0 & 0 \\
  0 & 0 & 0 & 0 & 0 \\
  0 & 0 & 0 & 0 & 0 \\
  0 & 0 & 0 & $\ddots$ & $\vdots$ \\
  0 & 0 & 0 & $\cdots$ & 0 \\
 };
 \matrix[auto matrix=x](matx){
  0 & 0 & 0 & $\cdots$ & 0 \\
  0 & 0 & 0 & $\cdots$ & 0 \\
  0 & 0 & 0 & $\cdots$ & 0 \\
  $\vdots$ & $\vdots$ & $\vdots$ & $\ddots$ & $\vdots$ \\
  0 & 0 & 0 & $\cdots$ & 0 \\
 };
 \draw[thick,-stealth] ([xshift=1ex]matx.south east) -- ([xshift=1ex]matz.south east)
  node[midway,below] {$C$};
 \draw[thick,-stealth] ([yshift=-1ex]matx.south west) --
  ([yshift=-1ex]matx.south east) node[midway,below] {size};
 \draw[thick,-stealth] ([xshift=-1ex]matx.north west)
   -- ([xshift=-1ex]matx.south west) node[midway,above,rotate=90] {mineral};
\end{tikzpicture}

    \caption{Representation of empty mcg 3D matrix}
    \label{fig:mcg_matrix}
\end{figure}


% Figure of mono- and polycrystalline grains %

The different mineral classes are assigned a number from 0 to m, with m being the total number of mineral classes. %
An array of numbers from 0 to m in SedGen not only represents the mineral occurrences (number frequencies) but also the interfacial composition. %
The interfacial composition is represented by the order of the crystals in an array in which the array represents one poly-crystalline grain. %
It should be noted that at the start of a SedGen initialization there is thus only one poly-crystalline `grain', the block of parent rock from which all other grains will derive. %
As weathering kicks in, the initial array will be split into arrays of smaller length and when this length becomes one, a mono-crystalline grain (mcg) is formed. %
When this happens, the mcg is transferred from the list of poly-crystalline grains (pcg) to a matrix of mcg with dimensions m x d where d is the number of grain size classes. %
The grain size representation is done in a similar way as the mineral occurrence/interface array, except for the fact that d numbers are used instead of m. %
Each number ranging from 0 to d thus represents a size/volume class. %

% Figure of interface and grain size representation arrays %

When intra-crystal breakage kicks in, mcg formed during inter-crystal breakage, will break. %
To restrict numerical operations for this to happen the number of ways a mcg might be broken is limited in relation to d, i.e. the amount of discretization of the model. %
Since a geometric series of grain size classes is used the ratio between neighbouring size classes is constant. %
This property might be exploited to determine a dictionary of size classes that `belong together' during the initialization phase of SedGen. %

% Example of search_bins and search_bins_ratio approach %



\section{Weathering modules}
    A distinction has to be made between mechanical and chemical weathering as to how they interact on the building blocks of the model.
    Mechanical weathering will operate on the interfaces of grains, or in the case of intra-crystal breakage of mcg, on its virtual interfaces. %
    In the case of inter-crystal breakage, upon breakage along an interface, this interface is thus destroyed and removed from the model`s interface counts matrix. %

    Chemical weathering will operate on the crystals themselves and this may result in two cases depending on the crystal`s size and chemical weathering state.
    \begin{enumerate}
        \item When a crystal is chemically weathered and its size is still bigger than its chemical weathering state`s residue threshold, it is kept in-place and only residue is formed due to weathering process.
        \item When a crystal is chemically weathered and its size is smaller than its chemical weathering state`s residue threshold, the crystal as a whole is dissolved and thus added to residue (of the corresponding mineral class). %
    \end{enumerate}
    \subsection{Mechanical weathering}
        \subsubsection{Inter-crystal breakage}

        \subsubsection{Intra-crystal breakage}

    \subsection{Chemical weathering}
    To accommodate the arithmetic nature of chemical weathering, a new approach to the binning was added since the binning consists of a geometric series. %
    By using multiple arrays of bins with each array representing the initial geometric series of bins minus x times the chemical weathering, a new trade-off between accuracy and performance is obtained. %
    This operation results in the expansion of multiple matrices from 2D to 3D with the dimensions being n x m x d, where n is the number of iterations the model has gone through. %
    As chemical weathering occurs once every time step, n immediately reflects the number of times chemical weathering has occurred on a certain grain. %
    By thus shifting the grains to higher n arrays, chemical weathering is simulated. %
    By requesting the corresponding bin array, the exact bin values for a specific chemical weathered grain may be retrieved. %


        \subsubsection{Chemical weathering of mcg}
        % Weathering
        % Formation of residue due to dissolution of mcg
        % Formation of residue due to weathering

        \subsubsection{Chemical weathering of pcg}
        % Weathering
        % Formation of mcg due to dissolution of all but one grain in a pcg
        % Recalculation of interfaces

        % Formation of residue due to dissolution of pcg
        % Formation of residue due to weathering

        \subsubsection{Clay formation}

\section{Chapter contents}
\begin{itemize}
    \item General overview of how data should be formatted to be accepted by the model and the modules at work to clean, transform and calculate data in the case of legacy data.
    \item Types of weathering and their interaction
    \item Types of crystals/grains within SedGen (poly, mono, RF…)
    \item Discuss their implementation into the SedGen model and how they are tracked
    \item Different frameworks that are used in SedGen (e.g. Lagrangian framework, compositional space vs physical space)
    \item Discuss data that SedGen model can output and at what timesteps (GSD, mineralogy, grain number frequencies)

\end{itemize}



%%%%%%%%%%%%%%%%%%%%%%%%%%%%%%%%%%%%%%%%%%%%%%%%%%
% Keep the following \cleardoublepage at the end of this file,
% otherwise \includeonly includes empty pages.
\cleardoublepage

% vim: tw=70 nocindent expandtab foldmethod=marker foldmarker={{{}{,}{}}}
